\documentclass[12pt, letterpaper]{article}
\usepackage[spanish]{babel}
\usepackage{hyperref}
\author{CodeDude}
\title{Temario para el curso de ciencias de la computación (Introductivo)} 
\date{29 de Noviembre del 2020}
\begin{document}
\maketitle
\newpage
\begin{abstract}
Este documento presenta el temario para el curso de ciencias de la computación impartidas por codeDude, el cual funcionará como curso piloto o introducción
a las ciencias de la computación, por lo tanto este primer curso esta enfocado a personas que no saben de computación ni de  programación.
El curso tiene como objetivo, enseñar desde las bases de tópicos enfocados a las ciencias de la computación, hasta el punto en 
que los alumnos puedan desarrollarse en temas científicos
o en la industria enfocada a la informática.\\
El curso de ciencias de la computación tendrá como base la programación y las matemáticas enfocadas a la informática, desde un sencillo hola mundo
hasta complejos algoritmos que el alumno podra desarrollar por su cuenta. El alumno sera libre de elegir las herramientas (editores de texto, entornos de
desarrollo, lenguajes de programación) que él quiera entre las sugeridas por el tutor, si el alumno no sabe por donde empezar o que herramientas deba usar
el tutor le sugerirá las herramientas.\\
El curso esta planificado para los días Sábados y Domingos a las 11:00am donde el Sábado se enfocará a cuestiones técnicas del desarrollo de software y los
Domingos serán enfocados a las matemáticas.
\end{abstract}
\newpage
\section{Temario}
\begin{enumerate}
  \item Teoría de la informática.
    \begin{itemize}
      \item ¿Qué es la computadora?	
      \item Historía de la computación.
      \item ¿Por qué son importantes las matemáticas?
      \item ¿Por qué es importante la logíca?
      \item Filosofía en la computación.
      \item Ética en la computación.
    \end{itemize} 
  \item Pseudocódigo
    \begin{itemize}
      \item Expresiones de pseudocódigo.
      \item Expresiones matemáticas.
      \item Expresiones logícas
      \item Algoritmos.
      \item Decisiones.
      \item Ciclo for.
      \item Ciclo while.
      \item Ciclo do while.
      \item Ejercicios.
    \end{itemize}
  \item Introducción a las matemáticas para computación.
    \begin{itemize}
      \item Repaso del algebra.	
      \item Tablas de verdad.
      \item Introducción a la logíca
    \end{itemize}
  \item Programación.
    \begin{itemize}
      \item Ejercicios de programación.	
    \end{itemize}
\end{enumerate}
\section{Material}
Los materiales a utilizar son:
\begin{itemize}
  \item Computadora (No tablets, No smartphones).
  \item Pseint (\href{http://pseint.sourceforge.net/}{Descarga})
  \item Un editor de texto de preferencia Visual Studio Code (\href{https://code.visualstudio.com/download}{Descarga})
\end{itemize}
\section{Horarios y Duración}
Sábados y Domingos a las 11 de la mañana, con una duración por clase de do horas.
\section{Duración del curso}
El curso es meramente un piloto o una introducción a las ciencias de l computación, si ve animos de seguir con este curso, se escribirá otro curso
como continuación de este.
\end{document}
