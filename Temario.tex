\documentclass[12pt, letterpaper]{article}
\usepackage[spanish]{babel}
\author{CodeDude}
\title{Temario para el curso de ciencias de la computación (Introductivo)} 
\date{29 de Noviembre del 2020}
\begin{document}
\maketitle
\newpage
\begin{abstract}
Este documento presenta el temario para el curso de ciencias de la computación impartidas por codeDude, el cual funcionará como curso piloto o introducción
a las ciencias de la computación.
El curso tiene como objetivo, enseñar desde las bases de tópicos enfocados a las ciencias de la computación, hasta el punto en 
que los alumnos puedan desarrollarse en temas científicos
o en la industria enfocada a la informática.\\
El curso de ciencias de la computación tendrá como base la programación y las matemáticas enfocadas a la informática, desde un sencillo hola mundo
hasta complejos algoritmos que el alumno podra desarrollar por su cuenta. El alumno sera libre de elegir las herramientas (editores de texto, entornos de
desarrollo, lenguajes de programación) que él quiera entre las sugeridas por el tutor, si el alumno no sabe por donde empezar o que herramientas deba usar
el tutor le sugerirá las herramientas.\\
El curso esta planificado para los días Sábados y Domingos a las 11:00am donde el Sábado se enfocará a cuestiones técnicas del desarrollo de software y los
Domingos serán enfocados a las matemáticas.
\newpage
\section{Temario}
\begin{enumerate}
  \item Teoría de la informática.
    \begin{itemize}
      \item ¿Qué es la computadora?	
      \item Historía de la computación
      \item ¿Por qué son importantes las matemáticas?
      \item ¿Por qué es importante la logíca?
    \end{itemize} 
\end{enumerate}
\end{abstract}
\end{document}
